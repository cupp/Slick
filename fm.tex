\documentclass[11pt]{amsart}
\usepackage{times}
\usepackage{amssymb,latexsym}
\usepackage[usenames, dvipsnames]{color}
\usepackage{ wasysym }

\newcommand{\lgap}{12pt}                            % Line gap
\newcommand{\slgap}{4pt}                            % Small line gap
\newcommand{\equivs}{\ensuremath{\;\equiv\;}}       % Equivales with space
\newcommand{\equivss}{\ensuremath{\;\;\equiv\;\;}}  % Equivales with double space
\newcommand{\nequiv}{\ensuremath{\not\equiv}}       % Inequivalent
\newcommand{\impl}{\ensuremath{\Rightarrow}}        % Implies
\newcommand{\nimpl}{\ensuremath{\not\Rightarrow}}   % Does not imply
\newcommand{\foll}{\ensuremath{\Leftarrow}}         % Follows from
\newcommand{\nfoll}{\ensuremath{\not\Leftarrow}}    % Does not follow from
\newcommand{\proofbreak}{\\ \\ \\ \\}

% These macros are used for quantifications. Thanks to David Gries for sharing
\newcommand{\thedr}{\rule[-.25ex]{.32mm}{1.75ex}}   % Symbol that separates dummy from range in quantification
\newcommand{\dr}{\;\,\thedr\,\;}                    % Symbol that separates dummy from range, with spacing
\newcommand{\rb}{:}                                 % Symbol that separates range from body in quantification
\newcommand{\drrb}{\;\thedr\,{:}\;}                 % Symbol that separates dummy from body when range is missing
\newcommand{\all}{\forall}                          % Universal quantification
\newcommand{\ext}{\exists}                          % Existential quantification
\newcommand{\Gll} {\langle}                         % Open hint
\newcommand{\Ggg} {\rangle}                         % Close hint

% Proof
\newcommand{\Step}[1]{\>{$#1$}}
%\newcommand{\Hint}[1] {\\=\>\>\ \ \ $\Gll\ \mbox{#1}\ \Ggg$ \\}   % Single line hint
\newcommand{\Hint}[1] {\\=\>\>\ \ \ $\Gll$\ \text{#1}\ $\Ggg$ \\}   % Single line hint
\newcommand{\done}{{\color{BurntOrange} \ \ $//$}}

% Math symbols
\newcommand{\nat}{\mathbb{N}}
\newcommand{\real}{\mathbb{R}}
\newcommand{\integer}{\mathbb{Z}}
\newcommand{\bool}{\mathbb{B}}

% Single and double quotes
\newcommand{\Lq}{\mbox{`}}
\newcommand{\Rq}{\mbox{'}}
\newcommand{\Lqq}{\mbox{``}}
\newcommand{\Rqq}{\mbox{''}}


\oddsidemargin  0.0in
\evensidemargin 0.0in
\textwidth      6.5in
\headheight     0.0in
\topmargin      0.0in
\textheight=9.0in
\parindent=0in
\pagestyle{empty}

\begin{document}
\begin{tabbing}
99.\;\=(m)\;\=\kill
\underline{Exercise 8.5a}\\\\

\Step{(\Sigma i \dr 0 \leq i < n + 1 : b[i])}
\\$=$\>\>\ \ \ $\Gll$\ \text{ (8.23) Split off term }\ $\Ggg$ \\
\Step{b[0] + (\Sigma i \dr 0 < i < n + 1 : b[i])}
\done

\end{tabbing}
\newpage
\begin{tabbing}
99.\;\=(m)\;\=\kill

\underline{Exercise 8.5b}\\\\

\Step{(\Sigma i \dr 0 \leq i \leq n : b[i])}
\\$=$\>\>\ \ \ $\Gll$\ \text{ (8.23) Split off term }\ $\Ggg$ \\
\Step{(\Sigma i \dr 0 \leq i < n : b[i]) + b[n]}
\done

\end{tabbing}
\newpage
\begin{tabbing}
99.\;\=(m)\;\=\kill

\color{blue}Prove\ (8.23b) Split off term:\ \ $(\star i \mid 0 \le i < n + 1 : P) = P[i := 0] \star (\star i \mid 0 < i < n + 1: P)$\\
\\

\Step{(\star i \dr 0 \leq i < n + 1 : P)}
\\$=$\>\>\ \ \ $\Gll$\ \text{ 0 $\leq$ \textit{i} $<$ \textit{n} $+$ 1  $\equiv$  \textit{i} $=$ 0 $\vee$ 0 $<$ \textit{i} $<$ \textit{n} $+$ 1 }\ $\Ggg$ \\
\Step{(\star i \dr i = 0 \vee 0 < i < n + 1 : P)}
\\$=$\>\>\ \ \ $\Gll$\ \text{ (8.16) Range split }\ $\Ggg$ \\
\Step{(\star i \dr i = 0 : P) \star (\star i \dr 0 < i < n + 1 : P)}
\\$=$\>\>\ \ \ $\Gll$\ \text{ (8.14) One$-$point rule }\ $\Ggg$ \\
\Step{P[i := 0] \star (\star i \dr 0 < i < n + 1 : P)}
\done

\end{tabbing}
\newpage
\begin{tabbing}
99.\;\=(m)\;\=\kill

\color{blue}Prove\ $0 \leq i < n + 1 \equiv 0 \leq i < n \vee i = n$\\
\\

\Step{0 \leq i < n + 1}
\\$=$\>\>\ \ \ $\Gll$\ \text{ Remove the conjunctive abbreviation }\ $\Ggg$ \\
\Step{0 \leq i \wedge i < n + 1}
\\$=$\>\>\ \ \ $\Gll$\ \text{ \textit{i} $<$ \textit{n} $+$ 1  $\equiv$  \textit{i} $<$ \textit{n} $\vee$ \textit{i} $=$ \textit{n} }\ $\Ggg$ \\
\Step{0 \leq i \wedge (i < n \vee i = n)}
\\$=$\>\>\ \ \ $\Gll$\ \text{ (3.46) Distributivity of $\wedge$ over $\vee$ }\ $\Ggg$ \\
\Step{(0 \leq i \wedge i < n) \vee (0 \leq i \wedge i = n)}
\\$=$\>\>\ \ \ $\Gll$\ \text{ (3.84a) Substitution }\ $\Ggg$ \\
\Step{(0 \leq i \wedge i < n) \vee (0 \leq n \wedge i = n)}
\\$=$\>\>\ \ \ $\Gll$\ \text{ Assume 0 $\leq$ \textit{n} }\ $\Ggg$ \\
\Step{(0 \leq i \wedge i < n) \vee (true \wedge i = n)}
\\$=$\>\>\ \ \ $\Gll$\ \text{ (3.39) Identity of $\wedge$ }\ $\Ggg$ \\
\Step{(0 \leq i \wedge i < n) \vee i = n}
\\$=$\>\>\ \ \ $\Gll$\ \text{ Reintroduce the conjunctive meaning }\ $\Ggg$ \\
\Step{0 \leq i < n \vee i = n}
\done

\end{tabbing}\end{document}

