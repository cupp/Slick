\documentclass[11pt]{amsart}
\usepackage{times}
\usepackage{amssymb,latexsym}
\usepackage[usenames, dvipsnames]{color}
\usepackage{ wasysym }

\newcommand{\lgap}{12pt}                            % Line gap
\newcommand{\slgap}{4pt}                            % Small line gap
\newcommand{\equivs}{\ensuremath{\;\equiv\;}}       % Equivales with space
\newcommand{\equivss}{\ensuremath{\;\;\equiv\;\;}}  % Equivales with double space
\newcommand{\nequiv}{\ensuremath{\not\equiv}}       % Inequivalent
\newcommand{\impl}{\ensuremath{\Rightarrow}}        % Implies
\newcommand{\nimpl}{\ensuremath{\not\Rightarrow}}   % Does not imply
\newcommand{\foll}{\ensuremath{\Leftarrow}}         % Follows from
\newcommand{\nfoll}{\ensuremath{\not\Leftarrow}}    % Does not follow from
\newcommand{\proofbreak}{\\ \\ \\ \\}

% These macros are used for quantifications. Thanks to David Gries for sharing
\newcommand{\thedr}{\rule[-.25ex]{.32mm}{1.75ex}}   % Symbol that separates dummy from range in quantification
\newcommand{\dr}{\;\,\thedr\,\;}                    % Symbol that separates dummy from range, with spacing
\newcommand{\rb}{:}                                 % Symbol that separates range from body in quantification
\newcommand{\drrb}{\;\thedr\,{:}\;}                 % Symbol that separates dummy from body when range is missing
\newcommand{\all}{\forall}                          % Universal quantification
\newcommand{\ext}{\exists}                          % Existential quantification
\newcommand{\Gll} {\langle}                         % Open hint
\newcommand{\Ggg} {\rangle}                         % Close hint

% Proof
\newcommand{\Step}[1]{\>{$#1$}}
%\newcommand{\Hint}[1] {\\=\>\>\ \ \ $\Gll\ \mbox{#1}\ \Ggg$ \\}   % Single line hint
\newcommand{\Hint}[1] {\\=\>\>\ \ \ $\Gll$\ \text{#1}\ $\Ggg$ \\}   % Single line hint
\newcommand{\done}{{\color{BurntOrange} \ \ $//$}}

% Math symbols
\newcommand{\nat}{\mathbb{N}}
\newcommand{\real}{\mathbb{R}}
\newcommand{\integer}{\mathbb{Z}}
\newcommand{\bool}{\mathbb{B}}

% Single and double quotes
\newcommand{\Lq}{\mbox{`}}
\newcommand{\Rq}{\mbox{'}}
\newcommand{\Lqq}{\mbox{``}}
\newcommand{\Rqq}{\mbox{''}}


\oddsidemargin  0.0in
\evensidemargin 0.0in
\textwidth      6.5in
\headheight     0.0in
\topmargin      0.0in
\textheight=9.0in
\parindent=0in
\pagestyle{empty}

\begin{document}
\begin{tabbing}
99.\;\=(m)\;\=\kill
\color{blue}Prove\ (3.12) Double negation:\ \ $\neg \neg p \equiv p$\\
\color{blue}by showing equivalence to a previous theorem\\
\\

\underline{Proof}\\\\
\Step{\neg \neg p \equiv p}
\\$=$\>\>\ \ \ $\Gll$\ \text{ (3.11) $\neg$\textit{p} $\equiv$ \textit{q} $\equiv$ \textit{p} $\equiv$ $\neg$\textit{q}, with \textit{p}, \textit{q} $:=$ $\neg$\textit{p}, \textit{p} }\ $\Ggg$ \\
\Step{\neg p \equiv \neg p}
\\$=$\>\>\ \ \ $\Gll$\ \text{ Identity of $\equiv$ }\ $\Ggg$ \\
\Step{true}
\done

\end{tabbing}
\newpage
\begin{tabbing}
99.\;\=(m)\;\=\kill

\color{blue}Prove\ (3.13) Negation of \textit{false}:\ \ $\neg \textit{false} \equiv \textit{true}$\\
\color{blue}by showing equivalence to a previous theorem\\
\\

\underline{Proof}\\\\
\Step{\neg false}
\\$=$\>\>\ \ \ $\Gll$\ \text{ Definition of false (3.8) }\ $\Ggg$ \\
\Step{\neg \neg true}
\\$=$\>\>\ \ \ $\Gll$\ \text{ Double negation }\ $\Ggg$ \\
\Step{true}
\done

\end{tabbing}
\newpage
\begin{tabbing}
99.\;\=(m)\;\=\kill

\color{blue}Prove\ (3.14) :\ \ $(p \not \equiv q)\ \equiv\ \neg p \equiv q$\\
\color{blue}by showing equivalence to a previous theorem\\
\\

\underline{Proof}\\\\
\Step{p \not \equiv q}
\\$=$\>\>\ \ \ $\Gll$\ \text{ Definition of $\not \equiv$ (3.10) }\ $\Ggg$ \\
\Step{\neg (p \equiv q)}
\\$=$\>\>\ \ \ $\Gll$\ \text{ Distributivity of $\neg$ over $\equiv$ (3.9) }\ $\Ggg$ \\
\Step{\neg p \equiv q}
\done

\end{tabbing}
\newpage
\begin{tabbing}
99.\;\=(m)\;\=\kill

\color{blue}Prove\ (3.19) Mutual interchangeability:\ \ $p \not \equiv q \equiv r \ \ \equiv \ \ p \equiv q \not \equiv r$\\
\color{blue}by showing equivalence to a previous theorem\\
\\

\underline{Proof}\\\\
\Step{p \not \equiv q \equiv r}
\\$=$\>\>\ \ \ $\Gll$\ \text{ Definition of $\not \equiv$ (3.10) with \textit{p}, \textit{q} $:=$ \textit{p}, \textit{q} $\equiv$ r}\ $\Ggg$ \\
\Step{\neg (p \equiv q \equiv r)}
\\$=$\>\>\ \ \ $\Gll$\ \text{ Definition of $\not \equiv$ (3.10) with \textit{p}, \textit{q} $:=$ \textit{p} $\equiv$ \textit{q}, \textit{r} }\ $\Ggg$ \\
\Step{p \equiv q \not \equiv r}
\done

\end{tabbing}\end{document}

