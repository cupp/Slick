\documentclass[11pt]{amsart}
\usepackage{times}
\usepackage{amssymb,latexsym}
\usepackage[usenames, dvipsnames]{color}
\usepackage{ wasysym }

\newcommand{\lgap}{12pt}                            % Line gap
\newcommand{\slgap}{4pt}                            % Small line gap
\newcommand{\equivs}{\ensuremath{\;\equiv\;}}       % Equivales with space
\newcommand{\equivss}{\ensuremath{\;\;\equiv\;\;}}  % Equivales with double space
\newcommand{\nequiv}{\ensuremath{\not\equiv}}       % Inequivalent
\newcommand{\impl}{\ensuremath{\Rightarrow}}        % Implies
\newcommand{\nimpl}{\ensuremath{\not\Rightarrow}}   % Does not imply
\newcommand{\foll}{\ensuremath{\Leftarrow}}         % Follows from
\newcommand{\nfoll}{\ensuremath{\not\Leftarrow}}    % Does not follow from
\newcommand{\proofbreak}{\\ \\ \\ \\}

% These macros are used for quantifications. Thanks to David Gries for sharing
\newcommand{\thedr}{\rule[-.25ex]{.32mm}{1.75ex}}   % Symbol that separates dummy from range in quantification
\newcommand{\dr}{\;\,\thedr\,\;}                    % Symbol that separates dummy from range, with spacing
\newcommand{\rb}{:}                                 % Symbol that separates range from body in quantification
\newcommand{\drrb}{\;\thedr\,{:}\;}                 % Symbol that separates dummy from body when range is missing
\newcommand{\all}{\forall}                          % Universal quantification
\newcommand{\ext}{\exists}                          % Existential quantification
\newcommand{\Gll} {\langle}                         % Open hint
\newcommand{\Ggg} {\rangle}                         % Close hint

% Proof
\newcommand{\Step}[1]{\>{$#1$}}
%\newcommand{\Hint}[1] {\\=\>\>\ \ \ $\Gll\ \mbox{#1}\ \Ggg$ \\}   % Single line hint
\newcommand{\Hint}[1] {\\=\>\>\ \ \ $\Gll$\ \text{#1}\ $\Ggg$ \\}   % Single line hint
\newcommand{\done}{{\color{BurntOrange} \ \ $//$}}

% Math symbols
\newcommand{\nat}{\mathbb{N}}
\newcommand{\real}{\mathbb{R}}
\newcommand{\integer}{\mathbb{Z}}
\newcommand{\bool}{\mathbb{B}}

% Single and double quotes
\newcommand{\Lq}{\mbox{`}}
\newcommand{\Rq}{\mbox{'}}
\newcommand{\Lqq}{\mbox{``}}
\newcommand{\Rqq}{\mbox{''}}


\oddsidemargin  0.0in
\evensidemargin 0.0in
\textwidth      6.5in
\headheight     0.0in
\topmargin      0.0in
\textheight=9.0in
\parindent=0in
\pagestyle{empty}

\begin{document}
\begin{tabbing}
99.\;\=(m)\;\=\kill
\color{blue}Prove\ $(C \wedge T \Rightarrow S) \wedge S \Rightarrow (C \Rightarrow T)$\\

\\
\text{}\ \ \\
\text{\ \ We\ \ associate\ \ identifiers\ \ with\ \ the\ \ primitive\ \ propositions:}\ \ \\
\text{}\ \ \\
\text{\ \ \ \ \ \ \ \ \ \ \textit{C}\ \ :\ \ The\ \ initialization\ \ is\ \ correct,}\ \ \\
\text{\ \ \ \ \ \ \ \ \ \ \textit{T}\ \ :\ \ The\ \ loop\ \ terminates,}\ \ \\
\text{\ \ \ \ \ \ \ \ \ \ \textit{S}\ \ :\ \ \textit{P}\ \ is\ \ \textit{true}\ \ in\ \ the\ \ final\ \ state.}\ \ \\
\text{}\ \ \\
\text{\ \ The\ \ boolean\ \ expression\ \ is\ \ then}\ \ \\
\text{}\ \ \\
\text{\ \ \ \ \ \ \ \ \ \ (\textit{C}\ \ $\wedge$\ \ \textit{T}\ \ $\Rightarrow$\ \ \textit{S})\ \ $\wedge$\ \ \textit{S}\ \ $\Rightarrow$\ \ (\textit{C}\ \ $\Rightarrow$\ \ \textit{T})}\ \ \\
\text{}\ \ \\
\text{\ \ We\ \ calculate:}\ \ \\
\text{}\ \ \\
\Step{(C \wedge T \Rightarrow S) \wedge S}
\\\\$=$\>\>\ \ \ $\Gll$\ \text{ (3.59) \textit{p} $\Rightarrow$ \textit{q} $\equiv$ $\neg$\textit{p} $\vee$ \textit{q} }\ $\Ggg$ \\
\Step{((C \wedge T) \vee S) \wedge S}
\\$=$\>\>\ \ \ $\Gll$\ \text{ (3.43a) Absorption }\ $\Ggg$ \\
\Step{S}
\\
\text{}\ \ \\
\text{\ \ At\ \ this\ \ point,\ \ we\ \ do\ \ not\ \ believe\ \ that\ \ \textit{S}\ \ implies\ \ \textit{C}\ \ $\Rightarrow$\ \ \textit{T}.\ \ The\ \ expression}\ \ \\
\text{\ \ \textit{S}\ \ $\Rightarrow$\ \ (\textit{C}\ \ $\Rightarrow$\ \ \textit{T})\ \ gives\ \ us\ \ a\ \ hint\ \ on\ \ what\ \ assignment\ \ of\ \ values\ \ for\ \ \textit{C},\ \ \textit{T},\ \ and\ \ \textit{S}\ \ to}\ \ \\
\text{\ \ choose\ \ so\ \ that\ \ the\ \ original\ \ expression\ \ is\ \ \textit{false}:}\ \ \\
\text{}\ \ \\
\text{\ \ \ \ \ \ \ \ \textit{S},\ \ \textit{T},\ \ \textit{C}\ \ $:=$\ \ \textit{true},\ \ \textit{false},\ \ \textit{true}.}\ \ \\
\text{}\ \ \\
\text{\ \ Hence,\ \ the\ \ argument\ \ is\ \ invalid.}\ \ \\

\end{tabbing}
\newpage
\begin{tabbing}
99.\;\=(m)\;\=\kill

\color{blue}Reprove\ $p \wedge q \Rightarrow p \wedge (q \vee r)$\\

\color{blue}by assuming the conjuncts of the antecedent\\\\
\\\Step{p \wedge (q \vee r)}
\\$=$\>\>\ \ \ $\Gll$\ \text{ Assume conjunct \textit{p} }\ $\Ggg$ \\
\Step{true \wedge (q \vee r)}
\\$=$\>\>\ \ \ $\Gll$\ \text{ Identity of $\wedge$ }\ $\Ggg$ \\
\Step{q \vee r}
\\$=$\>\>\ \ \ $\Gll$\ \text{ Assume conjunct \textit{q} }\ $\Ggg$ \\
\Step{true \vee r}
\\$=$\>\>\ \ \ $\Gll$\ \text{ Zero of $\vee$ }\ $\Ggg$ \\
\Step{true}
\\\done

\end{tabbing}
\newpage
\begin{tabbing}
99.\;\=(m)\;\=\kill

\color{blue}Reprove\ (3.46) Distributivity of $\wedge$ over $\vee$:\ \ $p \wedge (q \vee r)\ \equiv\ (p \wedge q) \vee (p \wedge r)$\\
\color{blue}by case analysis on p\\ \\
Must prove\\\>(1)\>$true \wedge (q \vee r) \equiv (true \wedge q) \vee (true \wedge r)$\\\>(2)\>$false \wedge (q \vee r) \equiv (false \wedge q) \vee (false \wedge r)$\\\\\underline{Proof of (1)}\\
\\\Step{(true \wedge q) \vee (true \wedge r)}
\\$=$\>\>\ \ \ $\Gll$\ \text{ Identity of $\wedge$, twice }\ $\Ggg$ \\
\Step{q \vee r}
\\$=$\>\>\ \ \ $\Gll$\ \text{ Identity of $\wedge$ }\ $\Ggg$ \\
\Step{true \wedge (q \vee r)}
\\\done
\\\\\underline{Proof of (2)}\\
\\\Step{(false \wedge q) \vee (false \wedge r)}
\\$=$\>\>\ \ \ $\Gll$\ \text{ Zero of $\wedge$, twice }\ $\Ggg$ \\
\Step{false \vee false}
\\$=$\>\>\ \ \ $\Gll$\ \text{ Identity of $\vee$ }\ $\Ggg$ \\
\Step{false}
\\$=$\>\>\ \ \ $\Gll$\ \text{ Zero of $\wedge$ }\ $\Ggg$ \\
\Step{false \wedge (q \vee r)}
\\\done

\end{tabbing}\end{document}

