\documentclass[11pt]{amsart}
\usepackage{times}
\usepackage{amssymb,latexsym}
\usepackage[usenames, dvipsnames]{color}
\usepackage{ wasysym }

\newcommand{\lgap}{12pt}                            % Line gap
\newcommand{\slgap}{4pt}                            % Small line gap
\newcommand{\equivs}{\ensuremath{\;\equiv\;}}       % Equivales with space
\newcommand{\equivss}{\ensuremath{\;\;\equiv\;\;}}  % Equivales with double space
\newcommand{\nequiv}{\ensuremath{\not\equiv}}       % Inequivalent
\newcommand{\impl}{\ensuremath{\Rightarrow}}        % Implies
\newcommand{\nimpl}{\ensuremath{\not\Rightarrow}}   % Does not imply
\newcommand{\foll}{\ensuremath{\Leftarrow}}         % Follows from
\newcommand{\nfoll}{\ensuremath{\not\Leftarrow}}    % Does not follow from
\newcommand{\proofbreak}{\\ \\ \\ \\}

% These macros are used for quantifications. Thanks to David Gries for sharing
\newcommand{\thedr}{\rule[-.25ex]{.32mm}{1.75ex}}   % Symbol that separates dummy from range in quantification
\newcommand{\dr}{\;\,\thedr\,\;}                    % Symbol that separates dummy from range, with spacing
\newcommand{\rb}{:}                                 % Symbol that separates range from body in quantification
\newcommand{\drrb}{\;\thedr\,{:}\;}                 % Symbol that separates dummy from body when range is missing
\newcommand{\all}{\forall}                          % Universal quantification
\newcommand{\ext}{\exists}                          % Existential quantification
\newcommand{\Gll} {\langle}                         % Open hint
\newcommand{\Ggg} {\rangle}                         % Close hint

% Proof
\newcommand{\Step}[1]{\>{$#1$}}
%\newcommand{\Hint}[1] {\\=\>\>\ \ \ $\Gll\ \mbox{#1}\ \Ggg$ \\}   % Single line hint
\newcommand{\Hint}[1] {\\=\>\>\ \ \ \color{CadetBlue}$\Gll$\ \text{#1}\ $\Ggg$ \\}   % Single line hint
\newcommand{\done}{{\color{BurntOrange} \ \ $//$}}

% Math symbols
\newcommand{\nat}{\mathbb{N}}
\newcommand{\real}{\mathbb{R}}
\newcommand{\integer}{\mathbb{Z}}
\newcommand{\bool}{\mathbb{B}}

% Single and double quotes
\newcommand{\Lq}{\mbox{`}}
\newcommand{\Rq}{\mbox{'}}
\newcommand{\Lqq}{\mbox{``}}
\newcommand{\Rqq}{\mbox{''}}


\oddsidemargin  0.0in
\evensidemargin 0.0in
\textwidth      6.5in
\headheight     0.0in
\topmargin      0.0in
\textheight=9.0in
\parindent=0in
\pagestyle{empty}

\begin{document}
\begin{tabbing}
99.\;\=(m)\;\=\kill
Prove\ (3.61) \textbf{Contrapositive}:\ \ $p \Rightarrow q\ \equiv\ \neg q \Rightarrow \neg p$\\ \\

\Step{\neg q \Rightarrow \neg p}
\Hint{ (3.59) with p, q $:=$ $\neg$q, $\neg$p }

\Step{\neg \neg q \vee \neg p}
\Hint{ Double negation }

\Step{q \vee \neg p}
\Hint{ Symmetry of $\vee$ }

\Step{\neg p \vee q}
\Hint{ (3.59) }

\Step{p \Rightarrow q}
\done

\end{tabbing}
\newpage
\begin{tabbing}
99.\;\=(m)\;\=\kill

Prove\ (3.63) \textbf{Distributivity of $\Rightarrow$ over $\equiv$}:\ \ $p \Rightarrow (q \equiv r)\ \equiv\ (p \Rightarrow q)\ \equiv\ (p \Rightarrow r)$\\ \\

\Step{p \Rightarrow q \equiv p \Rightarrow r}
\Hint{ Implication (3.59), twice }

\Step{\neg p \vee q \equiv \neg p \vee r}
\Hint{ Distributivity of $\vee$ over $\equiv$ }

\Step{\neg p \vee (q \equiv r)}
\Hint{ Implication (3.59) }

\Step{p \Rightarrow (q \equiv r)}
\done

\end{tabbing}
\newpage
\begin{tabbing}
99.\;\=(m)\;\=\kill

Prove\ (3.65) \textbf{Shunting}:\ \ $p \wedge q \Rightarrow r\ \equiv\ p \Rightarrow (q \Rightarrow r)$\\ \\

\Step{p \wedge q \Rightarrow r}
\Hint{ Implication (3.59) }

\Step{\neg (p \wedge q) \vee r}
\Hint{ De Morgan }

\Step{\neg p \vee \neg q \vee r}
\Hint{ Associativity of $\vee$ }

\Step{\neg p \vee (\neg q \vee r)}
\Hint{ Implication (3.59) }

\Step{\neg p \vee (q \Rightarrow r)}
\Hint{ Implication (3.59) }

\Step{p \Rightarrow (q \Rightarrow r)}
\done

\end{tabbing}
\newpage
\begin{tabbing}
99.\;\=(m)\;\=\kill

Prove\ (3.70) \ \ $p \vee q \Rightarrow p \wedge q\ \equiv\ p\ \equiv\ q$\\ \\

\Step{p \vee q \Rightarrow p \wedge q}
\Hint{ Implication (3.59) }

\Step{\neg (p \vee q) \vee (p \wedge q)}
\Hint{ De Morgan }

\Step{(\neg p \wedge \neg q) \vee (p \wedge q)}
\Hint{ Alternate definition of $\equiv$ }

\Step{p \equiv q}
\done

\end{tabbing}
\newpage
\begin{tabbing}
99.\;\=(m)\;\=\kill

Prove\ (3.71) \textbf{Reflexivity of $\Rightarrow$}:\ \ $p \Rightarrow p$\\ \\

\Step{p \Rightarrow p}
\Hint{ Implication (3.59) }

\Step{\neg p \vee p}
\Hint{ Excluded middle }

\Step{true}
\done

\end{tabbing}
\newpage
\begin{tabbing}
99.\;\=(m)\;\=\kill

Prove\ (3.72) \textbf{Right zero of $\Rightarrow$}:\ \ $p \Rightarrow \textit{true}\ \equiv\ \textit{true}$\\ \\

\Step{p \Rightarrow true}
\Hint{ Implication (3.59) }

\Step{\neg p \vee true}
\Hint{ Zero of $\vee$ }

\Step{true}
\done

\end{tabbing}
\newpage
\begin{tabbing}
99.\;\=(m)\;\=\kill

Prove\ (3.76a) \textbf{Weakening/strengthening}:\ \ $p \Rightarrow p \vee q$\\ \\

\Step{p \Rightarrow p \vee q}
\Hint{ Implication (3.59) }

\Step{\neg p \vee p \vee q}
\Hint{ Excluded middle }

\Step{true \vee q}
\Hint{ Zero of $\vee$ }

\Step{true}
\done

\end{tabbing}
\newpage
\begin{tabbing}
99.\;\=(m)\;\=\kill

Prove\ (3.77) \textbf{Modus ponens}:\ \ $p \wedge (p \Rightarrow q) \Rightarrow q$\\ \\

\Step{p \wedge (p \Rightarrow q) \Rightarrow q}
\Hint{ (3.66) p $\wedge$ (p $\Rightarrow$ q) $\equiv$ p $\wedge$ q }

\Step{p \wedge q \Rightarrow q}
\Hint{ Strengthening (3.76b) }

\Step{true}
\done

\end{tabbing}\end{document}

