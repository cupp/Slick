\documentclass[11pt]{amsart}
\usepackage{times}
\usepackage{amssymb,latexsym}
\usepackage[usenames, dvipsnames]{color}
\usepackage{ wasysym }

\newcommand{\lgap}{12pt}                            % Line gap
\newcommand{\slgap}{4pt}                            % Small line gap
\newcommand{\equivs}{\ensuremath{\;\equiv\;}}       % Equivales with space
\newcommand{\equivss}{\ensuremath{\;\;\equiv\;\;}}  % Equivales with double space
\newcommand{\nequiv}{\ensuremath{\not\equiv}}       % Inequivalent
\newcommand{\impl}{\ensuremath{\Rightarrow}}        % Implies
\newcommand{\nimpl}{\ensuremath{\not\Rightarrow}}   % Does not imply
\newcommand{\foll}{\ensuremath{\Leftarrow}}         % Follows from
\newcommand{\nfoll}{\ensuremath{\not\Leftarrow}}    % Does not follow from
\newcommand{\proofbreak}{\\ \\ \\ \\}

% These macros are used for quantifications. Thanks to David Gries for sharing
\newcommand{\thedr}{\rule[-.25ex]{.32mm}{1.75ex}}   % Symbol that separates dummy from range in quantification
\newcommand{\dr}{\;\,\thedr\,\;}                    % Symbol that separates dummy from range, with spacing
\newcommand{\rb}{:}                                 % Symbol that separates range from body in quantification
\newcommand{\drrb}{\;\thedr\,{:}\;}                 % Symbol that separates dummy from body when range is missing
\newcommand{\all}{\forall}                          % Universal quantification
\newcommand{\ext}{\exists}                          % Existential quantification
\newcommand{\Gll} {\langle}                         % Open hint
\newcommand{\Ggg} {\rangle}                         % Close hint

% Proof
\newcommand{\Step}[1]{\>{$#1$}}
%\newcommand{\Hint}[1] {\\=\>\>\ \ \ $\Gll\ \mbox{#1}\ \Ggg$ \\}   % Single line hint
\newcommand{\Hint}[1] {\\=\>\>\ \ \ $\Gll$\ \text{#1}\ $\Ggg$ \\}   % Single line hint
\newcommand{\done}{{\color{BurntOrange} \ \ $//$}}

% Math symbols
\newcommand{\nat}{\mathbb{N}}
\newcommand{\real}{\mathbb{R}}
\newcommand{\integer}{\mathbb{Z}}
\newcommand{\bool}{\mathbb{B}}

% Single and double quotes
\newcommand{\Lq}{\mbox{`}}
\newcommand{\Rq}{\mbox{'}}
\newcommand{\Lqq}{\mbox{``}}
\newcommand{\Rqq}{\mbox{''}}


\oddsidemargin  0.0in
\evensidemargin 0.0in
\textwidth      6.5in
\headheight     0.0in
\topmargin      0.0in
\textheight=9.0in
\parindent=0in
\pagestyle{empty}

\begin{document}
\begin{tabbing}
99.\;\=(m)\;\=\kill
\color{blue}Prove\ (3.85a) Replace by $\textit{true}$:\ \ $p \Rightarrow E^z_p\ \equiv\ p \Rightarrow E^z_\textit{true}$\\
\color{blue}by showing the LHS is equivalent to the RHS\\
\\

\underline{Proof}\\\\
\Step{p \Rightarrow E^{z}_{p}}
\\$=$\>\>\ \ \ $\Gll$\ \text{ Identity of $\equiv$ }\ $\Ggg$ \\
\Step{(p \equiv true) \Rightarrow E^{z}_{p}}
\\$=$\>\>\ \ \ $\Gll$\ \text{ (3.84b) with \textit{e}, \textit{f} $:=$ \textit{p}, true }\ $\Ggg$ \\
\Step{(p \equiv true) \Rightarrow E^{z}_{true}}
\\$=$\>\>\ \ \ $\Gll$\ \text{ Identity of $\equiv$ }\ $\Ggg$ \\
\Step{p \Rightarrow E^{z}_{true}}
\done

\end{tabbing}
\newpage
\begin{tabbing}
99.\;\=(m)\;\=\kill

\color{blue}Prove\ (3.85b) Replace by $\textit{true}$:\ \ $q \wedge p \Rightarrow E^z_p\ \equiv\ q \wedge p \Rightarrow E^z_\textit{true}$\\
\color{blue}by showing the LHS is equivalent to the RHS\\
\\

\underline{Proof}\\\\
\Step{q \wedge p \Rightarrow E^{z}_{p}}
\\$=$\>\>\ \ \ $\Gll$\ \text{ Identity of $\equiv$ }\ $\Ggg$ \\
\Step{q \wedge (p \equiv true) \Rightarrow E^{z}_{p}}
\\$=$\>\>\ \ \ $\Gll$\ \text{ (3.84c) with \textit{e}, \textit{f} $:=$ \textit{p}, true }\ $\Ggg$ \\
\Step{q \wedge (p \equiv true) \Rightarrow E^{z}_{true}}
\\$=$\>\>\ \ \ $\Gll$\ \text{ Identity of $\equiv$ }\ $\Ggg$ \\
\Step{q \wedge p \Rightarrow E^{z}_{true}}
\done

\end{tabbing}
\newpage
\begin{tabbing}
99.\;\=(m)\;\=\kill

\color{blue}Prove\ (4.1) :\ \ $p \Rightarrow (q \Rightarrow p)$\\
\color{blue}by showing the RHS follows from the LHS\\
\\

\underline{Proof}\\\\
\Step{q \Rightarrow p}
\\$=$\>\>\ \ \ $\Gll$\ \text{ (3.59) }\ $\Ggg$ \\
\Step{\neg q \vee p}
\\$\Leftarrow$\>\>\ \ \ $\Gll$\ \text{ (3.76a) Strengthening }\ $\Ggg$ \\
\Step{p}
\done

\end{tabbing}
\newpage
\begin{tabbing}
99.\;\=(m)\;\=\kill

\color{blue}Prove\ (4.3) Monotonicity of $\wedge$:\ \ $(p \Rightarrow q) \Rightarrow (p \wedge r \Rightarrow q \wedge r)$\\
\color{blue}by showing the RHS follows from the LHS\\
\\

\underline{Proof}\\\\
\Step{p \wedge r \Rightarrow q \wedge r}
\\$=$\>\>\ \ \ $\Gll$\ \text{ (3.59) }\ $\Ggg$ \\
\Step{\neg (p \wedge r) \vee (q \wedge r)}
\\$=$\>\>\ \ \ $\Gll$\ \text{ De Morgan }\ $\Ggg$ \\
\Step{\neg p \vee \neg r \vee (q \wedge r)}
\\$=$\>\>\ \ \ $\Gll$\ \text{ Distbributivity of $\vee$ over $\wedge$ }\ $\Ggg$ \\
\Step{\neg p \vee ((\neg r \vee q) \wedge (\neg r \vee r))}
\\$=$\>\>\ \ \ $\Gll$\ \text{ Excluded middle }\ $\Ggg$ \\
\Step{\neg p \vee ((\neg r \vee q) \wedge true)}
\\$=$\>\>\ \ \ $\Gll$\ \text{ Identity of $\equiv$ }\ $\Ggg$ \\
\Step{\neg p \vee \neg r \vee q}
\\$\Leftarrow$\>\>\ \ \ $\Gll$\ \text{ (3.76a) Strengthening }\ $\Ggg$ \\
\Step{\neg p \vee q}
\\$=$\>\>\ \ \ $\Gll$\ \text{ (3.59) }\ $\Ggg$ \\
\Step{p \Rightarrow q}
\done

\end{tabbing}\end{document}

