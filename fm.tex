\documentclass[11pt]{amsart}
\usepackage{times}
\usepackage{amssymb,latexsym}
\usepackage[usenames, dvipsnames]{color}
\usepackage{ wasysym }

\newcommand{\lgap}{12pt}                            % Line gap
\newcommand{\slgap}{4pt}                            % Small line gap
\newcommand{\equivs}{\ensuremath{\;\equiv\;}}       % Equivales with space
\newcommand{\equivss}{\ensuremath{\;\;\equiv\;\;}}  % Equivales with double space
\newcommand{\nequiv}{\ensuremath{\not\equiv}}       % Inequivalent
\newcommand{\impl}{\ensuremath{\Rightarrow}}        % Implies
\newcommand{\nimpl}{\ensuremath{\not\Rightarrow}}   % Does not imply
\newcommand{\foll}{\ensuremath{\Leftarrow}}         % Follows from
\newcommand{\nfoll}{\ensuremath{\not\Leftarrow}}    % Does not follow from
\newcommand{\proofbreak}{\\ \\ \\ \\}

% These macros are used for quantifications. Thanks to David Gries for sharing
\newcommand{\thedr}{\rule[-.25ex]{.32mm}{1.75ex}}   % Symbol that separates dummy from range in quantification
\newcommand{\dr}{\;\,\thedr\,\;}                    % Symbol that separates dummy from range, with spacing
\newcommand{\rb}{:}                                 % Symbol that separates range from body in quantification
\newcommand{\drrb}{\;\thedr\,{:}\;}                 % Symbol that separates dummy from body when range is missing
\newcommand{\all}{\forall}                          % Universal quantification
\newcommand{\ext}{\exists}                          % Existential quantification
\newcommand{\Gll} {\langle}                         % Open hint
\newcommand{\Ggg} {\rangle}                         % Close hint

% Proof
\newcommand{\Step}[1]{\>{$#1$}}
%\newcommand{\Hint}[1] {\\=\>\>\ \ \ $\Gll\ \mbox{#1}\ \Ggg$ \\}   % Single line hint
\newcommand{\Hint}[1] {\\=\>\>\ \ \ $\Gll$\ \text{#1}\ $\Ggg$ \\}   % Single line hint
\newcommand{\done}{{\color{BurntOrange} \ \ $//$}}

% Math symbols
\newcommand{\nat}{\mathbb{N}}
\newcommand{\real}{\mathbb{R}}
\newcommand{\integer}{\mathbb{Z}}
\newcommand{\bool}{\mathbb{B}}

% Single and double quotes
\newcommand{\Lq}{\mbox{`}}
\newcommand{\Rq}{\mbox{'}}
\newcommand{\Lqq}{\mbox{``}}
\newcommand{\Rqq}{\mbox{''}}


\oddsidemargin  0.0in
\evensidemargin 0.0in
\textwidth      6.5in
\headheight     0.0in
\topmargin      0.0in
\textheight=9.0in
\parindent=0in
\pagestyle{empty}

\begin{document}
\begin{tabbing}
99.\;\=(m)\;\=\kill
\\
\text{}\ \ \\
\text{5.1a)}\ \ \\
\text{}\ \ \\
\text{Let\ \ \textit{T}\ \ represent\ \ the\ \ program\ \ terminating}\ \ \\
\text{Let\ \ \textit{N}\ \ represent\ \ variable\ \ \textit{n}\ \ becoming\ \ 0}\ \ \\
\text{Let\ \ \textit{M}\ \ represent\ \ variable\ \ \textit{m}\ \ becoming\ \ 0}\ \ \\
\text{}\ \ \\
\text{The\ \ argument\ \ is\ \ ($\neg$\textit{T}\ \ $\vee$\ \ \textit{N})\ \ $\wedge$\ \ (\textit{N}\ \ $\Rightarrow$\ \ \textit{M})\ \ $\wedge$\ \ \textit{T}\ \ \ \ $\Rightarrow$\ \ \ \ \textit{M}}\ \ \\
\text{}\ \ \\
\text{The\ \ argument\ \ is\ \ a\ \ theorem\ \ by\ \ the\ \ following\ \ proof.}\ \ \\
\\\Step{(\neg T \vee N) \wedge (N \Rightarrow M) \wedge T}
\\$=$\>\>\ \ \ $\Gll$\ \text{ Absorption (3.44a) }\ $\Ggg$ \\
\Step{N \wedge (N \Rightarrow M) \wedge T}
\\$=$\>\>\ \ \ $\Gll$\ \text{ (3.66) }\ $\Ggg$ \\
\Step{N \wedge M \wedge T}
\\$\Rightarrow$\>\>\ \ \ $\Gll$\ \text{ (3.76b) Strengthening }\ $\Ggg$ \\
\Step{M}
\done

\end{tabbing}
\newpage
\begin{tabbing}
99.\;\=(m)\;\=\kill

\\
\text{}\ \ \\
\text{5.1c)}\ \ \\
\text{}\ \ \\
\text{Let\ \ \textit{M}\ \ represent\ \ a\ \ man\ \ being\ \ on\ \ the\ \ moon.}\ \ \\
\text{Let\ \ \textit{C}\ \ represent\ \ the\ \ moon\ \ being\ \ made\ \ of\ \ cheese.}\ \ \\
\text{Let\ \ \textit{I}\ \ represent\ \ the\ \ statement\ \ "\textit{I}\ \ am\ \ a\ \ monkey."}\ \ \\
\text{}\ \ \\
\text{The\ \ argument\ \ is\ \ (\textit{M}\ \ $\Rightarrow$\ \ \textit{C})\ \ $\wedge$\ \ (\textit{C}\ \ $\Rightarrow$\ \ \textit{I})\ \ $\wedge$\ \ ($\neg$\textit{M}\ \ $\vee$\ \ $\neg$\textit{C})\ \ \ \ $\Rightarrow$\ \ \ \ $\neg$\textit{C}\ \ $\vee$\ \ \textit{I}}\ \ \\
\text{}\ \ \\
\text{The\ \ argument\ \ is\ \ a\ \ theorem\ \ by\ \ the\ \ following\ \ proof.}\ \ \\
\\\Step{(M \Rightarrow C) \wedge (C \Rightarrow I) \wedge (\neg M \vee \neg C)}
\\$\Rightarrow$\>\>\ \ \ $\Gll$\ \text{ Strengthening (3.76b) }\ $\Ggg$ \\
\Step{C \Rightarrow I}
\\$=$\>\>\ \ \ $\Gll$\ \text{ (3.59) Implication }\ $\Ggg$ \\
\Step{\neg C \vee I}
\done

\end{tabbing}
\newpage
\begin{tabbing}
99.\;\=(m)\;\=\kill

\\
\text{}\ \ \\
\text{5.1d)}\ \ \\
\text{}\ \ \\
\text{Let\ \ \textit{J}\ \ represent\ \ Joe\ \ loves\ \ Mary.}\ \ \\
\text{Let\ \ \textit{M}\ \ represent\ \ Mom\ \ is\ \ mad.}\ \ \\
\text{Let\ \ \textit{F}\ \ represent\ \ Father\ \ is\ \ sad.}\ \ \\
\text{}\ \ \\
\text{The\ \ argument\ \ is\ \ (\textit{J}\ \ $\Rightarrow$\ \ \textit{M}\ \ $\vee$\ \ \textit{F})\ \ $\wedge$\ \ \textit{F}\ \ \ \ $\Rightarrow$\ \ \ \ (\textit{M}\ \ $\Rightarrow$\ \ $\neg$\textit{J})}\ \ \\
\text{}\ \ \\
\text{The\ \ argument\ \ is\ \ not\ \ a\ \ theorem\ \ by\ \ counterexample\ \ (\textit{J},\ \ \textit{true}),\ \ (\textit{M},\ \ \textit{true}),\ \ (\textit{F},\ \ \textit{true})}\ \ \\
\\\Step{(true \Rightarrow true \vee true) \wedge true \Rightarrow (true \Rightarrow \neg true)}
\\$=$\>\>\ \ \ $\Gll$\ \text{ zero of $\vee$, $\neg$true $\equiv$ false }\ $\Ggg$ \\
\Step{(true \Rightarrow true) \wedge true \Rightarrow (true \Rightarrow false)}
\\$=$\>\>\ \ \ $\Gll$\ \text{ true $\Rightarrow$ \textit{p} $\equiv$ \textit{p}, twice }\ $\Ggg$ \\
\Step{true \wedge true \Rightarrow false}
\\$=$\>\>\ \ \ $\Gll$\ \text{ \textit{p} $\Rightarrow$ false $\equiv$ false }\ $\Ggg$ \\
\Step{false}
\done

\end{tabbing}
\newpage
\begin{tabbing}
99.\;\=(m)\;\=\kill

\color{blue}Reprove\ (3.47a) De Morgan:\ \ $\neg (p \wedge q)\ \equiv\ \neg p \vee \neg q$\\

\color{blue}by contradiction\\\\
\\\Step{\neg (\neg (p \wedge q) \equiv \neg p \vee \neg q)}
\\$=$\>\>\ \ \ $\Gll$\ \text{ Distributivity of $\neg$ over $\equiv$ }\ $\Ggg$ \\
\Step{\neg \neg (p \wedge q) \equiv \neg p \vee \neg q}
\\$=$\>\>\ \ \ $\Gll$\ \text{ Double negation }\ $\Ggg$ \\
\Step{p \wedge q \equiv \neg p \vee \neg q}
\\$=$\>\>\ \ \ $\Gll$\ \text{ Golden rule }\ $\Ggg$ \\
\Step{p \equiv q \equiv p \vee q \equiv \neg p \vee \neg q}
\\$=$\>\>\ \ \ $\Gll$\ \text{ (3.32) }\ $\Ggg$ \\
\Step{q \equiv p \vee \neg q \equiv \neg p \vee \neg q}
\\$=$\>\>\ \ \ $\Gll$\ \text{ Distributivity of $\vee$ over $\equiv$ }\ $\Ggg$ \\
\Step{q \equiv (p \equiv \neg p) \vee \neg q}
\\$=$\>\>\ \ \ $\Gll$\ \text{ (3.15) }\ $\Ggg$ \\
\Step{q \equiv false \vee \neg q}
\\$=$\>\>\ \ \ $\Gll$\ \text{ Identity of $\vee$ }\ $\Ggg$ \\
\Step{q \equiv \neg q}
\\$=$\>\>\ \ \ $\Gll$\ \text{ (3.15) }\ $\Ggg$ \\
\Step{false}
\done

\end{tabbing}
\newpage
\begin{tabbing}
99.\;\=(m)\;\=\kill

\color{blue}Reprove\ (3.76c) Weakening/strengthening:\ \ $p \wedge q \Rightarrow p \vee q$\\

\color{blue}by proving the contrapositive: $\neg (p \vee q) \Rightarrow \neg (p \wedge q)$\\\\
\\\Step{\neg (p \vee q)}
\\$=$\>\>\ \ \ $\Gll$\ \text{ De Morgan }\ $\Ggg$ \\
\Step{\neg p \wedge \neg q}
\\$\Rightarrow$\>\>\ \ \ $\Gll$\ \text{ (3.76b) }\ $\Ggg$ \\
\Step{\neg p}
\\$\Rightarrow$\>\>\ \ \ $\Gll$\ \text{ (3.76a) }\ $\Ggg$ \\
\Step{\neg p \vee \neg q}
\\$=$\>\>\ \ \ $\Gll$\ \text{ De Morgan }\ $\Ggg$ \\
\Step{\neg (p \wedge q)}
\done

\end{tabbing}\end{document}

